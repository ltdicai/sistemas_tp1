\subsection{Lottery Scheduling}

La idea detr\'as de $lottery$ $scheduling$\cite{SchedLottery} es aplicar un enfoque probabil\'istico y pseudoaleatorio a la hora de asignar recursos a las tareas. Para lograr esto se utiliza un sistema de $tickets$ que representan derechos de acceso a recursos; las tareas poseen una cantidad determinada de $tickets$, puediendo ser distinta de una tarea a otra y cada vez que el $scheduler$ decide a qu\'e tarea otorgar los recursos lo hace mediante una loter\'ia, elige uno entre todos los $tickets$ del sistema. El proceso que tenga ese $ticket$ ser\'a el ganador y se le asignar\'a el recurso que est\'a demandado.

Los $tickets$ son una forma de cuantificar los derechos de acceso a recursos; cuantos m\'as $tickets$ tenga una tarea, m\'as probable es que gane la loter\'ia. Tambi\'en permiten homogeneizar los diferentes tipos de recursos ya que todos son representados por los mismos tipos de $tickets$, esto aporta flexibilidad a la hora de asignarlos.

\subsubsection{Loter\'ias}

En cada $tick$ del reloj, el $scheduler$ toma una decisi\'on: si el proceso en ejecuci\'on agot\'o todo su quantum asignado, realiz\'o una llamada bloqueante o simplemente termin\'o exitosamente entonces se lleva a cabo una loter\'ia. Est\'a loter\'ia decide qui\'en va a ser el pr\'oximo proceso en utilizar el recurso deseado. Elegir el $ticket$ ganador se realiza mediante un algoritmo pseudoaleatorio que lo elige de manera uniforme, esto es, todos los tickets tienen una probabilidad igual de ganar la loter\'ia, a saber:

Sea $n$ la cantidad de $tickets$ y \{t$_i$\}$_{1 \leq i \leq n}$ el conjunto de ellos.

\[
	P(t_{i}) = \frac{1}{n}
\]

Luego, la probabilidad que un proceso $j$ que posee $t$ cantidad de $tickets$ gane la loter\'ia es igual a:


\[
	P(j \text{ gane la loter\'ia}) = \sum_{1}^{t}\frac{1}{n} = \frac{t}{n}
\]

Se puede probar que la cantidad de loter\'ias ganadas por la tarea $j$ sigue una distribuci\'on binomial de par\'ametro $p = \frac{t}{n}$. De esta manera se observa que la esperanza de la cantidad de loter\'ias ganadas por una tarea es igual a:

\[
	E(\text{cantidad de loter\'ias ganadas}) = \frac{t}{n}*\text{cantidad de loter\'ias realizadas}
\]

Esto implica que el valor promedio de la cantidad de loter\'ias ganadas es directamente proporcional a la cantidad de $tickets$ que posee la tarea. Por esta raz\'on se dice que $lottery$ $scheduling$ es probabil\'isticamente justo.

\subsubsection{Fairness}

\subsubsection{Compensation tickets}
