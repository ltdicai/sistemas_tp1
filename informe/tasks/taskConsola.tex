\subsection{TaskConsola}

\begin{algorithm}
 \caption{TaskConsola}
 \begin{algorithmic}[1]
 \Procedure{TaskConsola}{$cant\_ bloqueos, bmin, bmax$}
   \State{seteo semilla de random}
   \For{$i\gets 0, cant\_ bloqueos$}
     \State{$random\_ number\gets modulo(rand(),\ bmax - bmin + 1) + bmin$}\Comment{[1]}
     \State{$Uso\_ IO(random\_ number)$}    
   \EndFor
 \EndProcedure
 \end{algorithmic}
\end{algorithm}

[Nota 1]: \funcName{rand}() retorna un número pseudoaleatorio arbitrariamente entre 0 y $RAND\_MAX$, una constante del lenguaje. Vale entonces que si:

\begin{align*}
	\funcName{rand}() \in [0..RAND\_MAX]	&\implies \funcName{rand}()\text{ \textbf{m\'od} }(bmax-bmin+1) \in [0..bmax-bmin]\\
								&\implies \funcName{rand}()\text{ \textbf{m\'od} }(bmax-bmin+1) + bmin \in [bmin..bmax]
\end{align*}
